%%%%%%%%%%%%%%%%%%%%%%%%%%%%%%%%%%%%%%%%%
%
% CMPT 435
% Spring 2019
% Assignment Three
%
%%%%%%%%%%%%%%%%%%%%%%%%%%%%%%%%%%%%%%%%%

%%%%%%%%%%%%%%%%%%%%%%%%%%%%%%%%%%%%%%%%%
% Short Sectioned Assignment
% LaTeX Template
% Version 1.0 (5/5/12)
%
% This template has been downloaded from: http://www.LaTeXTemplates.com
% Original author: % Frits Wenneker (http://www.howtotex.com)
% License: CC BY-NC-SA 3.0 (http://creativecommons.org/licenses/by-nc-sa/3.0/)
% Modified by Alan G. Labouseur  - alan@labouseur.com
% Further Modified by Timothy M. Polizzi - Timpolizzi2@gmail.com
% Code Listings by LaTeX
%
%%%%%%%%%%%%%%%%%%%%%%%%%%%%%%%%%%%%%%%%%

%----------------------------------------------------------------
%	PACKAGES AND OTHER DOCUMENT CONFIGURATIONS
%----------------------------------------------------------------

\documentclass[letterpaper, 10pt]{article} 

\usepackage[english]{babel} % English language/hyphenation
\usepackage{graphicx}
\usepackage[lined,linesnumbered,commentsnumbered]{algorithm2e}
\usepackage{listings}
\usepackage{fancyhdr} % Custom headers and footers
\pagestyle{fancyplain} % Makes all pages in the document conform to the custom headers and footers
\usepackage{lastpage}
\usepackage{url}
\usepackage{listings}
\usepackage{color}

\definecolor{codegreen}{rgb}{0,0.6,0}
\definecolor{codegray}{rgb}{0.5,0.5,0.5}
\definecolor{codepurple}{rgb}{0.58,0,0.82}
\definecolor{backcolour}{rgb}{0.95,0.95,0.92}

\lstdefinestyle{mystyle}{
    backgroundcolor=\color{backcolour},   
    commentstyle=\color{codegreen},
    keywordstyle=\color{magenta},
    numberstyle=\tiny\color{codegray},
    stringstyle=\color{codepurple},
    basicstyle=\footnotesize,
    breakatwhitespace=false,         
    breaklines=true,                 
    captionpos=b,                    
    keepspaces=true,                 
    numbers=left,                    
    numbersep=5pt,                  
    showspaces=false,                
    showstringspaces=false,
    showtabs=false,                  
    tabsize=2
}

\lstset{style=mystyle}

\fancyhead{} 
% No page header - if you want one, create it in the same way as the footers below
\fancyfoot[L]{} % Empty left footer
\fancyfoot[C]{page \thepage\ of \pageref{LastPage}}
% Page numbering for center footer
\fancyfoot[R]{}

\renewcommand{\headrulewidth}{0pt} % Remove header underlines
\renewcommand{\footrulewidth}{0pt} % Remove footer underlines
\setlength{\headheight}{5pt} % Customize the height of the header

%----------------------------------------------------------------
%	TITLE SECTION
%----------------------------------------------------------------

% Create horizontal rule command with 1 argument of height
\newcommand{\horrule}[1]{\rule{\linewidth}{#1}} 

\title{	
   \normalfont \normalsize 
   \textsc{CMPT 435 - Spring 2019 - Dr. Labouseur} \\[10pt] % Header stuff.
   \horrule{0.5pt} \\[0.25cm] 	% Top horizontal rule
   \huge Assignment Three -- Graphs and Trees\\     	    % Assignment title
   \horrule{0.5pt} \\[0.25cm] 	% Bottom horizontal rule
}

\author{Timothy Polizzi \\ \normalsize Timothy.Polizzi1@Marist.edu}

\date{\normalsize\today} 	% Today's date.

\begin{document}

\maketitle % Print the title

%----------------------------------------------------------------
%   CONTENT SECTION
%----------------------------------------------------------------

\noindent

\section{Graphs}
In this assignment we were tasked with the implementation of a graph in three ways: through means of a matrix or a two dimensional array; an adjacency list; and a number of linked nodes. We were then tasked with analyzing the asymptotic running time of these traversals.

\subsection{Traversals}
\begin{tabular}{lll}
\underline{ Traversal } & \underline{Asymptotic Running time}\\
Depth First     & $\Theta(|E| + |V|)$\\
Breadth First   & $\Theta(|E| + |V|)$\\
\end{tabular}

\subsection{Explanation}
For a depth first traversal or a breadth first traversal, it is actually only ever at most O$(|E| + |V|)$, where E is the edges visited and V is the vertices. This complexity is actually quite simple as you will only ever visit an edge twice and never a vertex more than once. Because it is theoretically possible for a small number of edges to be connected to a large number of vertices and vice versa, it also means that whichever there are more of (edges or vertices) will dominate the time complexity.

\section{Trees}
Besides for the graphs, there was also the binary search tree included in the project. This tree is constructed based on the rule set that all smaller items than the current node are to the left of it and all larger are to the right. This makes it incredibly fast to find a specific item in, in fact a average run-time of O($\log n$). This also speeds up adding items and removing them as you will always only follow down a direct path, and meaning the worst case will leave a user at O(n) time which is still not bad. In my code specifically, I was averaging about 20 comparisons a look-up in a binary search tree of 666 items.

\end{document}
