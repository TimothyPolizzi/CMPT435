%%%%%%%%%%%%%%%%%%%%%%%%%%%%%%%%%%%%%%%%%
%
% CMPT 435
% Spring 2019
% Assignment Three
%
%%%%%%%%%%%%%%%%%%%%%%%%%%%%%%%%%%%%%%%%%

%%%%%%%%%%%%%%%%%%%%%%%%%%%%%%%%%%%%%%%%%
% Short Sectioned Assignment
% LaTeX Template
% Version 1.0 (5/5/12)
%
% This template has been downloaded from: http://www.LaTeXTemplates.com
% Original author: % Frits Wenneker (http://www.howtotex.com)
% License: CC BY-NC-SA 3.0 (http://creativecommons.org/licenses/by-nc-sa/3.0/)
% Modified by Alan G. Labouseur  - alan@labouseur.com
% Further Modified by Timothy M. Polizzi - Timpolizzi2@gmail.com
% Code Listings by LaTeX
%
%%%%%%%%%%%%%%%%%%%%%%%%%%%%%%%%%%%%%%%%%

%----------------------------------------------------------------
%	PACKAGES AND OTHER DOCUMENT CONFIGURATIONS
%----------------------------------------------------------------

\documentclass[letterpaper, 10pt]{article} 

\usepackage[english]{babel} % English language/hyphenation
\usepackage{graphicx}
\usepackage[lined,linesnumbered,commentsnumbered]{algorithm2e}
\usepackage{listings}
\usepackage{fancyhdr} % Custom headers and footers
\pagestyle{fancyplain} % Makes all pages in the document conform to the custom headers and footers
\usepackage{lastpage}
\usepackage{url}
\usepackage{listings}
\usepackage{color}

\definecolor{codegreen}{rgb}{0,0.6,0}
\definecolor{codegray}{rgb}{0.5,0.5,0.5}
\definecolor{codepurple}{rgb}{0.58,0,0.82}
\definecolor{backcolour}{rgb}{0.95,0.95,0.92}

\lstdefinestyle{mystyle}{
    backgroundcolor=\color{backcolour},   
    commentstyle=\color{codegreen},
    keywordstyle=\color{magenta},
    numberstyle=\tiny\color{codegray},
    stringstyle=\color{codepurple},
    basicstyle=\footnotesize,
    breakatwhitespace=false,         
    breaklines=true,                 
    captionpos=b,                    
    keepspaces=true,                 
    numbers=left,                    
    numbersep=5pt,                  
    showspaces=false,                
    showstringspaces=false,
    showtabs=false,                  
    tabsize=2
}

\lstset{style=mystyle}

\fancyhead{} 
% No page header - if you want one, create it in the same way as the footers below
\fancyfoot[L]{} % Empty left footer
\fancyfoot[C]{page \thepage\ of \pageref{LastPage}}
% Page numbering for center footer
\fancyfoot[R]{}

\renewcommand{\headrulewidth}{0pt} % Remove header underlines
\renewcommand{\footrulewidth}{0pt} % Remove footer underlines
\setlength{\headheight}{5pt} % Customize the height of the header

%----------------------------------------------------------------
%	TITLE SECTION
%----------------------------------------------------------------

% Create horizontal rule command with 1 argument of height
\newcommand{\horrule}[1]{\rule{\linewidth}{#1}} 

\title{	
   \normalfont \normalsize 
   \textsc{CMPT 435 - Spring 2019 - Dr. Labouseur} \\[10pt] % Header stuff.
   \horrule{0.5pt} \\[0.25cm] 	% Top horizontal rule
   \huge Assignment Four -- Directions and Spice\\     	    % Assignment title
   \horrule{0.5pt} \\[0.25cm] 	% Bottom horizontal rule
}

\author{Timothy Polizzi \\ \normalsize Timothy.Polizzi1@Marist.edu}

\date{\normalsize\today} 	% Today's date.

\begin{document}

\maketitle % Print the title

%----------------------------------------------------------------
%   CONTENT SECTION
%----------------------------------------------------------------

\noindent

\section{Bellman-Ford Single Source Shortest Path}
The Bellman-Ford Single Source Shortest Path is an algorithm that is based off of Dijkstra's, where you give a source node in a graph and the shortest path to all nodes is listed. The difference between Dijkstra's and Bellman-Ford's is that Bellman-Ford's provides a way to check if there is an infinite negative-weight cycle which would mean any point on the graph can be traversed to in negative infinity weight.

\subsection{Asymptotic Analysis}
As the Bellman-Ford approach to the Single Source Shortest Path problem was based off of Dijkstra's as an attempt to fix an edge case, it does cost a bit more to do so. Specifically the time of the worst case becomes $\Theta(|E| |V|)$ while Dijkstra's is only $\Theta(|E| + |V| log |V|)$. On the other hand at best case the Bellman-Ford algorithm can be run in only $\Theta(|E|)$, as it can theoretically run once through the edges if all the edges are in maximum to minimum order. The worst case on the other hand is due to the possibility for a need each edge for each vertex.

\section{Fractional Knapsack}
Fractional Knapsack is a different take on the knapsack problem then 0-1 knapsack, where you take an item and split it into a knapsack. This allows for the method of greedy programming to work, and is more efficient.

\subsection{Asymptotic Analysis}
The weighted knapsack problem is able to be solved in $\theta(n)$ time which is due to only needing to be run through its list of items to be put into the bag.

\end{document}
